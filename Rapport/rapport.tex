\documentclass[a4paper, titlepage]{report}

\usepackage[utf8]{inputenc} % accents
\usepackage[T1]{fontenc}      % caractères français
\usepackage{geometry}         % marges
\usepackage[francais]{babel}  % langue
\usepackage{graphicx}         % images
\usepackage{verbatim}         % texte préformaté
\usepackage{listings}
\usepackage{moreverb}

\title{Rapport - Stéganographie}      % renseigne le titre
\author{Jezequel Corentin - Saingre Dimitri - Beaulieu Loic - Caillaud Pierre-Antoine}%""l'auteur
\date{16 Janvien 2015}           %   "   "   la future date de parution

\pagestyle{headings}          % affiche un rappel discret (en haut à gauche)
                              % de la partie dans laquel on se situe
\setlength{\parindent}{0cm}
\setlength{\parskip}{1ex plus 0.5ex minus 0.2ex}
\newcommand{\hsp}{\hspace{20pt}}
\newcommand{\HRule}{\rule{\linewidth}{0.5mm}}

\begin{document}

\begin{titlepage}
  \begin{sffamily}
  \begin{center}

    % Upper part of the page. The '~' is needed because \\
    % only works if a paragraph has started.
    \includegraphics[scale=1]{univnantes.png}~\\[1.5cm]

    \textsc{\LARGE IUT de Nantes}\\[2cm]

    \textsc{\LargeRapport Stéganographie}\\[1.5cm]

    % Title
    \HRule \\[0.4cm]
    { \huge \bfseries Projet \\[0.4cm] }

    \HRule \\[2cm]
    
    \\[2cm]

    % Author and supervisor
    \begin{minipage}{0.4\textwidth}
      \begin{flushleft} \large
        Dimitri Saingre\\
        Corentin Jezequel\\
        Beaulieu Loic\\
        Pierre-Antoine Caillaud\\
        Info 2 Groupe 3\\
      \end{flushleft}
    \end{minipage}
    \begin{minipage}{0.4\textwidth}
      \begin{flushleft} \large
        \emph{Professeur :} S. Faucou
      \end{flushleft}
    \end{minipage}

    \vfill

    % Bottom of the page
    {\large 16 janvier 2015}


  \end{center}
  \end{sffamily}
\end{titlepage}

\chapter*{Introduction}

  \section*{Définition de la stéganogaphie}

    \paragraph{La stéganographie est l'art de la dissimulation, au contraire de la cryptographie qui consiste à rendre un message inintelligible. Le but de la stéganographie est de dissimuler un message dans un autre message. Pour exemple, la stéganographie consisterait à enterrer son argent dans son jardin, alors que la cryptographie consisterait à mettre son argent dans un coffre fort}

  \section{Présentation du projet}

    \paragraph{Nous avons plusieurs tâches à réaliser pour mener ce projet à bien. Nous avons eu pour consigne : \newline - Semestre 3 : Réalisation d'un logiciel de stéganographie offrant le choix parmi plusieurs (au moins deux) algorithmes \newline - Semestre 4 : Portage du logiciel sur Android  Développement d'un module de chiffrement symétrique (AES) au sein du logiciel. \newline Au moment où ce rapport est écrit, nous sommes au semestre 3 et nous allons faire une présentation de ce qui a été fait pour le moment.}

  \section{Division du travail}

    \paragraph{Le travail du semestre 3 a été diviser en deux. Deux équipes ont été formées. \newline Loic et Pierre-Antoine : conception d'un programme en Java permettant de cacher du texte dans une image. \newline Dimitri et Corentin : à compléter \newline \newline Nous allons commencer par vous présenter le programme permettant de cacher du texte dans des images.}


\chapter{}


\end{document}


